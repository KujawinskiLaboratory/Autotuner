\nonstopmode{}
\documentclass[a4paper]{book}
\usepackage[times,inconsolata,hyper]{Rd}
\usepackage{makeidx}
\usepackage[utf8]{inputenc} % @SET ENCODING@
% \usepackage{graphicx} % @USE GRAPHICX@
\makeindex{}
\begin{document}
\chapter*{}
\begin{center}
{\textbf{\huge Package `Autotuner'}}
\par\bigskip{\large \today}
\end{center}
\begin{description}
\raggedright{}
\inputencoding{utf8}
\item[Type]\AsIs{Package}
\item[Title]\AsIs{What the Package Does (Title Case)}
\item[Version]\AsIs{0.1.0}
\item[Author]\AsIs{Craig McLean}
\item[Maintainer]\AsIs{The package maintainer }\email{yourself@somewhere.net}\AsIs{}
\item[Description]\AsIs{More about what it does (maybe more than one line)
Use four spaces when indenting paragraphs within the Description.}
\item[License]\AsIs{MIT + file LICENSE}
\item[Encoding]\AsIs{UTF-8}
\item[Depends]\AsIs{R (>= 3.4)}
\item[LazyData]\AsIs{true}
\item[RoxygenNote]\AsIs{6.1.1}
\item[Imports]\AsIs{MSnbase, DT, devtools, shinyFiles, xcms, RColorBrewer, shiny,
PTXQC, dplyr, rlang, plyr, mzR, assertthat, scales, methods,
entropy, cluster, mmetspData}
\item[Suggests]\AsIs{testthat (>= 2.1.0), covr}
\end{description}
\Rdcontents{\R{} topics documented:}
\inputencoding{utf8}
\HeaderA{Autotuner-class}{Autotuner}{Autotuner.Rdash.class}
\aliasA{Autotuner}{Autotuner-class}{Autotuner}
%
\begin{Description}\relax
This file contains the skeleton to the Autotuner class used
through out Autoutuner.

This object is a generic object designed to run the different
functions of the ms2sweeper package. The slots represent content or data
that the package uses throughout the different functions.
\end{Description}
%
\begin{Section}{Slots}

\begin{description}

\item[\code{time}] - A list containing vectors of scan time points from each sample.

\item[\code{intensity}] - A list containing vectors of scan intensity points from
each sample.

\item[\code{peaks}] - Regions within each sample identified as peaks by sliding
window analysis.

\item[\code{peak\_table}] - A data.frame containing information on each peak after
further processing is done to the data.

\item[\code{peak\_difference}] - A data.frame containing information on how peaks
are eluted differently over time.

\item[\code{metadata}] - A data.frame containing metadata for all samples to be
run on Autotuner.

\item[\code{file\_paths}] - A string path that leads to the samples to be run on
Autotuner.

\item[\code{file\_col}] - A string for the column name of the column within the
metadata that has specific sample names.

\item[\code{factorCol}] - A string for the column name of the column within the
metadata that has specific sample class names.

\end{description}
\end{Section}
\inputencoding{utf8}
\HeaderA{checkBounds}{checkBounds}{checkBounds}
%
\begin{Description}\relax
Recursive function used to find how far a binned feature might
extend beyond the boundary of the originally defined TIC peak.
\end{Description}
%
\begin{Usage}
\begin{verbatim}
checkBounds(mass, upper = T, mzDb, currentIndex, intensityStorage,
  ppmEst, scans, origBound, header)
\end{verbatim}
\end{Usage}
%
\begin{Arguments}
\begin{ldescription}
\item[\code{mass}] - Specific mass being checked against adjacent scans.

\item[\code{upper}] - A boolean value that tells the algorithm to check indices
greater than the entered one.  If false, it will check values less thant the
entered one.

\item[\code{mzDb}] - A list of data.frames containing the m/z and intensity values
from each scan's mass spectra.

\item[\code{currentIndex}] - Numerical index indicating which scan contains
feature specific information.

\item[\code{intensityStorage}] - comming soon

\item[\code{ppmEst}] - Scalar numerical value meant to represent the ppm of the
instrument.

\item[\code{scans}] - Set of all possible ms1 scans for the sample.

\item[\code{origBound}] - The original scan bound location of the peak.

\item[\code{header}] - A data.fame containing metadata on the sample like
spectra type (MS1 vs MS2), retention time, and scan count.
\end{ldescription}
\end{Arguments}
%
\begin{Value}
This function returns the last index the feature is detected.
\end{Value}
\inputencoding{utf8}
\HeaderA{checkEICPeaks}{checkEICPeaks}{checkEICPeaks}
%
\begin{Description}\relax
This function is the outer most function used to check for
individual EIC peak specific parameters.
\end{Description}
%
\begin{Usage}
\begin{verbatim}
checkEICPeaks(mzDb, header, observedPeak, massThresh = 0.005, useGap,
  varExpThresh, returnPpmPlots, plotDir, filename)
\end{verbatim}
\end{Usage}
%
\begin{Arguments}
\begin{ldescription}
\item[\code{mzDb}] - A list of data.frames containing the m/z and intensity values
from each scan's mass spectra.

\item[\code{header}] - A data.fame containing metadata on the sample like
spectra type, retention time, and scan count.

\item[\code{observedPeak}] - A list with names 'start' and 'end' containing
scalar values representing the calculated peak boundary points

\item[\code{massThresh}] - A generous exact mass error threshold used to estimate
PPM for features.

\item[\code{useGap}] - Parameter carried into checkEICPeaks that tells Autotuner
whether to use the gap statustic to determine the proper number of clusters
to use during ppm parameter estimation.

\item[\code{varExpThresh}] - Numeric value representing the variance explained
threshold to use if useGap is false.

\item[\code{returnPpmPlots}] - Boolean value that tells R to return plots for
ppm distributions.

\item[\code{plotDir}] - Path where to store plots.

\item[\code{filename}] - A string containing the name of the current data file being
analyzed.
\end{ldescription}
\end{Arguments}
\inputencoding{utf8}
\HeaderA{countMaxima}{countMaxima}{countMaxima}
%
\begin{Description}\relax
This function is designed to determine the maximum number of
scans of the longest peak in the sample. It does this using the signal
processing function used earlier. This function is used during the estimation
of the maximum peakwdith.
\end{Description}
%
\begin{Usage}
\begin{verbatim}
countMaxima(peakBounds, sortedAllEIC, prevRange)
\end{verbatim}
\end{Usage}
%
\begin{Arguments}
\begin{ldescription}
\item[\code{peakBounds}] - What are the bounds of the peak.

\item[\code{sortedAllEIC}] - Original data object.

\item[\code{prevRange}] - Longest measured peak range.
\end{ldescription}
\end{Arguments}
%
\begin{Value}
maxScans - the maximum number of scans across all peaks
\end{Value}
\inputencoding{utf8}
\HeaderA{createAutotuner}{createAutotuner}{createAutotuner}
%
\begin{Description}\relax
This function will create a Autotuner used to extract ms2s.
\end{Description}
%
\begin{Usage}
\begin{verbatim}
createAutotuner(data_paths, runfile, file_col, factorCol)
\end{verbatim}
\end{Usage}
%
\begin{Arguments}
\begin{ldescription}
\item[\code{data\_paths}] - A string path pointing at data files to load in Autotuner.

\item[\code{runfile}] - a data.frame of sample metadata.

\item[\code{file\_col}] - Character string of the column name of the column within
the runfile that contains sample names.

\item[\code{factorCol}] - Character string of the column name of the column within
the runfile that contains sample type factor.
\end{ldescription}
\end{Arguments}
\inputencoding{utf8}
\HeaderA{dissectScans}{dissectScans}{dissectScans}
%
\begin{Description}\relax
This function is designed to extract all MS1 scan features
observed within the bounds of the current TIC peak.
\end{Description}
%
\begin{Usage}
\begin{verbatim}
dissectScans(mzDb, observedPeak, header)
\end{verbatim}
\end{Usage}
%
\begin{Arguments}
\begin{ldescription}
\item[\code{mzDb}] - This is a list of two column data frames containing information
on each mass spectra within the data.

\item[\code{observedPeak}] - A list with 'start' and 'stop' boundaries of the
current peak.

\item[\code{header}] - This is the header file containing all the metadata
for the currently loaded sample.
\end{ldescription}
\end{Arguments}
\inputencoding{utf8}
\HeaderA{EICparams}{EICparams}{EICparams}
%
\begin{Description}\relax
This function is designed to calculate the recommended
parameters from EIC peaks. It is the main holder function for a lot of
different ones involved in calculating EIC parameters.
\end{Description}
%
\begin{Usage}
\begin{verbatim}
EICparams(Autotuner, massThresh, peak_table, useGap = T,
  varExpThresh = 0.8, returnPpmPlots = T, plotDir = ".",
  verbose = T)
\end{verbatim}
\end{Usage}
%
\begin{Arguments}
\begin{ldescription}
\item[\code{Autotuner}] - An Autotuner objected containing sample specific raw
data.

\item[\code{massThresh}] - A generous exact mass error threshold used to estimate
PPM for features.

\item[\code{peak\_table}] - table of peak width values extracted with the function
peak\_width\_table.

\item[\code{useGap}] - Parameter carried into checkEICPeaks that tells Autotuner
whether to use the gap statustic to determine the proper number of clusters
to use during ppm parameter estimation.

\item[\code{varExpThresh}] - Numeric value representing the variance explained
threshold to use if useGap is false.

\item[\code{returnPpmPlots}] - Boolean value that tells R to return plots for
ppm distributions.

\item[\code{plotDir}] - Path where to store plots.

\item[\code{verbose}] - Boolean value used to indicate whether checkEICPeaks function
returns messages to the console.
\end{ldescription}
\end{Arguments}
%
\begin{Details}\relax
The function CheckEICPeaks handles all the peak specific
computations.
\end{Details}
\inputencoding{utf8}
\HeaderA{estimatePPM}{estimatePPM}{estimatePPM}
%
\begin{Description}\relax
This function provides a convenient way to measure ppm between
two exact masses.
\end{Description}
%
\begin{Usage}
\begin{verbatim}
estimatePPM(first, second)
\end{verbatim}
\end{Usage}
%
\begin{Arguments}
\begin{ldescription}
\item[\code{first}] - first mass entered

\item[\code{second}] - second mass entered
\end{ldescription}
\end{Arguments}
%
\begin{Value}
The ppm error between the first and second entered mass
\end{Value}
\inputencoding{utf8}
\HeaderA{estimateSNThresh}{estimateSNThresh}{estimateSNThresh}
%
\begin{Description}\relax
This function is responsible for computing an estimated s/n
threshold.
\end{Description}
%
\begin{Usage}
\begin{verbatim}
estimateSNThresh(no_match, sortedAllEIC, approvedPeaks)
\end{verbatim}
\end{Usage}
%
\begin{Arguments}
\begin{ldescription}
\item[\code{no\_match}] - This is a vector of numerical indicies within the raw data
mapping to scan data considered to come from noise.

\item[\code{sortedAllEIC}] - This is the raw data from a single TIC peak.

\item[\code{approvedPeaks}] - This is a data.frame that contains information on
which peaks come from TIC data.
\end{ldescription}
\end{Arguments}
%
\begin{Value}
returns an estimated s/n threshold value
\end{Value}
\inputencoding{utf8}
\HeaderA{estimate\_loess}{estimate\_loess}{estimate.Rul.loess}
%
\begin{Description}\relax
estimate\_loess
\end{Description}
%
\begin{Usage}
\begin{verbatim}
estimate_loess(xs, max_missing, sampleClasses, threshold)
\end{verbatim}
\end{Usage}
%
\begin{Arguments}
\begin{ldescription}
\item[\code{xs}] - an xcmsSet object following peakpicking and grouping

\item[\code{max\_missing}] - Estimate for maximum missing peaks within a group to
be considered for retention time correction.

\item[\code{sampleClasses}] - String vector representing the classes each sample
belogns to.

\item[\code{threshold}] - peak width parameter to estimate retention time
correction does not overfit the data.
\end{ldescription}
\end{Arguments}
%
\begin{Value}
This function will return the maximum span that will still
provide retention time correction across features over time scales
as large as the estimated peak dispersion parameters.
\end{Value}
\inputencoding{utf8}
\HeaderA{estimate\_threshold}{estimate\_threshold}{estimate.Rul.threshold}
%
\begin{Description}\relax
At a high level, this function is used to estimate appropriate
thresholds to required to distinguish between different peaks accross
samples, and replicated ones that only vary due to retention time deviation.
It assumes that duplicate peaks are a relatively short distance from one
another relative to those that distinct.
\end{Description}
%
\begin{Usage}
\begin{verbatim}
estimate_threshold(distance_vector)
\end{verbatim}
\end{Usage}
%
\begin{Arguments}
\begin{ldescription}
\item[\code{distance\_vector}] - This vector contains the retention time differences
between individual peaks identified within a single sample TIC.
\end{ldescription}
\end{Arguments}
%
\begin{Details}\relax
To estimate threshold, all the measured distances are clustered
using k-means clustering with an itteritatively increasing number of
clusters. The itteration is complete once there are as many clusters as
peaks or when the median of the smallest distance cluster is larger than the
difference between the minimum and maximum distance values within that
cluster.
\end{Details}
%
\begin{Value}
A scalar value made up of the largest value predicted to come from
peaks identical peaks plus 1.
\end{Value}
\inputencoding{utf8}
\HeaderA{exportPeaks}{exportPeaks}{exportPeaks}
%
\begin{Description}\relax
This function contains the server component of the export Peak
portion of the shiny app used within Autotuner. This is the part where a
spreadsheet is saved as output.
\end{Description}
%
\begin{Usage}
\begin{verbatim}
exportPeaks(input, output, session, filledPeaks)
\end{verbatim}
\end{Usage}
%
\begin{Arguments}
\begin{ldescription}
\item[\code{input}] - List of UI entered values.

\item[\code{output}] - Values exported from server to the UI after computation.

\item[\code{session}] - String used to match the namespace of the UI and server

\item[\code{filledPeaks}] - The xcms object to get exported as a feature table.
\end{ldescription}
\end{Arguments}
\inputencoding{utf8}
\HeaderA{exportPeaksUI}{exportPeaksUI}{exportPeaksUI}
%
\begin{Description}\relax
This function contains the UI component of the export Peak
portion of the shiny app used within Autotuner. This is the part where a
spreadsheet is saved as output.
\end{Description}
%
\begin{Usage}
\begin{verbatim}
exportPeaksUI(id)
\end{verbatim}
\end{Usage}
%
\begin{Arguments}
\begin{ldescription}
\item[\code{id}] - character string used to map UI and server within App
\end{ldescription}
\end{Arguments}
\inputencoding{utf8}
\HeaderA{extract\_peaks}{extract\_peaks}{extract.Rul.peaks}
%
\begin{Description}\relax
This function is designed to extract peaks observed within the
TIC from each sample, and return their indicies for further processing.
\end{Description}
%
\begin{Usage}
\begin{verbatim}
extract_peaks(Autotuner, returned_peaks = 10, signals)
\end{verbatim}
\end{Usage}
%
\begin{Arguments}
\begin{ldescription}
\item[\code{Autotuner}] - An Autotuner objected containing sample specific raw
data.

\item[\code{returned\_peaks}] - A scalar number of peaks to return for visual
inspection. Five is the minimum possible value.
is the standard.

\item[\code{signals}] - List containing traces and locations where signals are
detected across all samples being checked by the algorithm.
\end{ldescription}
\end{Arguments}
%
\begin{Value}
peak\_table\_list - a list of data.frame tables containing information
on where each where peaks are located within each sample.
\end{Value}
\inputencoding{utf8}
\HeaderA{filterPeaksfromNoise}{filterPeaksfromNoise}{filterPeaksfromNoise}
%
\begin{Description}\relax
This function is desgined to perform the binning of potential
features. At this point in the algorithm, a potential feature is 2+ m/z
values that are within the generous exact mass error windown provided by
the user.
\end{Description}
%
\begin{Usage}
\begin{verbatim}
filterPeaksfromNoise(matchedMasses)
\end{verbatim}
\end{Usage}
%
\begin{Arguments}
\begin{ldescription}
\item[\code{matchedMasses}] - A data.frame containing information on the
retained bins.
\end{ldescription}
\end{Arguments}
%
\begin{Value}
A list with entries for noise peaks and true peaks.
\end{Value}
\inputencoding{utf8}
\HeaderA{filterPpmError}{filterPpmError}{filterPpmError}
%
\begin{Description}\relax
This function computes an estimate for the ppm error threshold.
\end{Description}
%
\begin{Usage}
\begin{verbatim}
filterPpmError(approvedPeaks, useGap, varExpThresh, returnPpmPlots,
  plotDir, observedPeak, filename)
\end{verbatim}
\end{Usage}
%
\begin{Arguments}
\begin{ldescription}
\item[\code{approvedPeaks}] - This is a data.frame with information on bins retained
after filtering with user input mz error threshold and continuity checks.

\item[\code{useGap}] - Parameter carried into checkEICPeaks that tells Autotuner
whether to use the gap statustic to determine the proper number of clusters
to use during ppm parameter estimation.

\item[\code{varExpThresh}] - Numeric value representing the variance explained
threshold to use if useGap is false.

\item[\code{returnPpmPlots}] - Boolean value that tells R to return plots for
ppm distributions.

\item[\code{plotDir}] - Path where to store plots.

\item[\code{observedPeak}] - A list with names 'start' and 'end' containing
scalar values representing the calculated peak boundary points

\item[\code{filename}] - A string containing the name of the current data file being
analyzed.
\end{ldescription}
\end{Arguments}
%
\begin{Details}\relax
A distribution is created from the set of all ppm values identified.
The most dense peak of this distribution is assumed to represent the standard
ppm error of the data.
\end{Details}
%
\begin{Value}
This function returns a scalar value representing ppm error estimate.
\end{Value}
\inputencoding{utf8}
\HeaderA{findPeakWidth}{findPeakWidth}{findPeakWidth}
%
\begin{Description}\relax
This function is designed to find the maximum peakwidth of an EIC observed within a
given TIC peak. It does so by using checkBounds to estimate width in time of a peak and countMaxima
to determine if a peak may be made up from two similar structural isomers.
\end{Description}
%
\begin{Usage}
\begin{verbatim}
findPeakWidth(approvScorePeaks, mzDb, header, sortedAllEIC, boundaries,
  ppmEst)
\end{verbatim}
\end{Usage}
%
\begin{Arguments}
\begin{ldescription}
\item[\code{approvScorePeaks}] - A data.frame containing information on the
retained bins.

\item[\code{mzDb}] - A list of data.frames containing the m/z and intensity values
from each scan's mass spectra.

\item[\code{header}] - A data.fame containing metadata on the sample like
spectra type (MS1 vs MS2), retention time, and scan count.

\item[\code{sortedAllEIC}] - a da

\item[\code{boundaries}] - A numeric vector with indicies representing the scans
bounding the original TIC peak.

\item[\code{ppmEst}] - A scalar value representing the calculated ppm error
used to generate data.
\end{ldescription}
\end{Arguments}
%
\begin{Value}
This function returns a scalar value representing an estimate for
the maximal peak width across samples.
\end{Value}
\inputencoding{utf8}
\HeaderA{findTruePeaks}{findTruePeaks}{findTruePeaks}
%
\begin{Description}\relax
This function is designed to filter out bins that don't come from
continuous scans. The idea is that after this stage, the data is ready for
parameter estimation.
\end{Description}
%
\begin{Usage}
\begin{verbatim}
findTruePeaks(truePeaks, sortedAllEIC)
\end{verbatim}
\end{Usage}
%
\begin{Arguments}
\begin{ldescription}
\item[\code{truePeaks}] - A lsit containing indicies representing each bin.

\item[\code{sortedAllEIC}] - All the raw ms1 data extracted from the EIC peak.
\end{ldescription}
\end{Arguments}
\inputencoding{utf8}
\HeaderA{find\_noise}{find\_noise}{find.Rul.noise}
%
\begin{Description}\relax
find\_noise
\end{Description}
%
\begin{Usage}
\begin{verbatim}
find_noise(time_trace)
\end{verbatim}
\end{Usage}
%
\begin{Arguments}
\begin{ldescription}
\item[\code{time\_trace}] - chromatographic intensity trace
\end{ldescription}
\end{Arguments}
%
\begin{Value}
This function determines the indexes within a chromatographic intensity trace
that are not noise. It does this by asking if a given point is greater than the average
of that point and the two bounding points on either side of it.
\end{Value}
\inputencoding{utf8}
\HeaderA{fineParams}{fineParams}{fineParams}
%
\begin{Description}\relax
This function contains the server component of the EIC based
parameter selection portion of the shiny app used within Autotuner. This
module is specifically involved in presentation of recommended EIC peaks
following the completion of Autotuner. It can help the user export a
the calculated parameters as a spreadsheet.
\end{Description}
%
\begin{Usage}
\begin{verbatim}
fineParams(input, output, session, params, runfile, Autotuner)
\end{verbatim}
\end{Usage}
%
\begin{Arguments}
\begin{ldescription}
\item[\code{input}] - List of UI entered values.

\item[\code{output}] - Values exported from server to the UI after computation.

\item[\code{session}] - String used to match the namespace of the UI and server

\item[\code{params}] - The reactive object containing all calculated parameter
estimates from the data.

\item[\code{runfile}] - A data.frame containing metadata for all the samples within
the selected folder.

\item[\code{Autotuner}] - Data with specifics on samples path and metadata.
\end{ldescription}
\end{Arguments}
\inputencoding{utf8}
\HeaderA{fineParamsUI}{fineParamsUI}{fineParamsUI}
%
\begin{Description}\relax
This function contains the UI component of the EIC based
parameter selection portion of the shiny app used within Autotuner.
\end{Description}
%
\begin{Usage}
\begin{verbatim}
fineParamsUI(id)
\end{verbatim}
\end{Usage}
%
\begin{Arguments}
\begin{ldescription}
\item[\code{id}] - character string used to map UI and server within App
\end{ldescription}
\end{Arguments}
\inputencoding{utf8}
\HeaderA{initialize,Autotuner-method}{Update an \code{\LinkA{Autotuner}{Autotuner.Rdash.class}} object}{initialize,Autotuner.Rdash.method}
%
\begin{Description}\relax
This method updates an \code{\LinkA{Autotuner}{Autotuner.Rdash.class}}
object to the latest definition.
\end{Description}
%
\begin{Usage}
\begin{verbatim}
## S4 method for signature 'Autotuner'
initialize(.Object, data_paths, runfile, file_col,
  factorCol)
\end{verbatim}
\end{Usage}
%
\begin{Arguments}
\begin{ldescription}
\item[\code{.Object}] - the \code{\LinkA{Autotuner}{Autotuner.Rdash.class}} object to update.

\item[\code{data\_paths}] - A string path pointing at data files to load in Autotuner.

\item[\code{runfile}] - a data.frame of sample metadata.

\item[\code{file\_col}] - Character string of the column name of the column within
the runfile that contains sample names.

\item[\code{factorCol}] - Character string of the column name of the column within
the runfile that contains sample type factor.
\end{ldescription}
\end{Arguments}
%
\begin{Value}
An updated \code{\LinkA{Autotuner}{Autotuner.Rdash.class}} containing all data from
the input object.
\end{Value}
%
\begin{Author}\relax
Craig McLean
\end{Author}
\inputencoding{utf8}
\HeaderA{launchAutotuner}{launchAutotuner}{launchAutotuner}
%
\begin{Description}\relax
This function deploys the Autotuner shiny app.
\end{Description}
%
\begin{Usage}
\begin{verbatim}
launchAutotuner()
\end{verbatim}
\end{Usage}
\inputencoding{utf8}
\HeaderA{loadData}{loadData}{loadData}
%
\begin{Description}\relax
This function contains the server component of the data loading
portion of the shiny app used within Autotuner.
\end{Description}
%
\begin{Usage}
\begin{verbatim}
loadData(input, output, session)
\end{verbatim}
\end{Usage}
%
\begin{Arguments}
\begin{ldescription}
\item[\code{input}] - List of UI entered values.

\item[\code{output}] - Values exported from server to the UI after computation.

\item[\code{session}] - String used to match the namespace of the UI and server
\end{ldescription}
\end{Arguments}
\inputencoding{utf8}
\HeaderA{loadDataUI}{loadDataUI}{loadDataUI}
%
\begin{Description}\relax
This function contains the UI component of the data loading
portion of the shiny app used within Autotuner.
\end{Description}
%
\begin{Usage}
\begin{verbatim}
loadDataUI(id)
\end{verbatim}
\end{Usage}
%
\begin{Arguments}
\begin{ldescription}
\item[\code{id}] - character string used to map UI and server within App'
\end{ldescription}
\end{Arguments}
\inputencoding{utf8}
\HeaderA{peakVis}{peakVis}{peakVis}
%
\begin{Description}\relax
This function contains the server component of the peak
visualization portion of the shiny app used within Autotuner. In this section
the user selects if signal processing parameters were properly tuned to
indentify individual TIC peaks.
\end{Description}
%
\begin{Usage}
\begin{verbatim}
peakVis(input, output, session, signalData, Autotuner)
\end{verbatim}
\end{Usage}
%
\begin{Arguments}
\begin{ldescription}
\item[\code{input}] - List of UI entered values.

\item[\code{output}] - Values exported from server to the UI after computation.

\item[\code{session}] - String used to match the namespace of the UI and server

\item[\code{signalData}] - Information on peaks selected from sliding windown
analysis in the previous page of shiny app.

\item[\code{Autotuner}] - Data with specifics on samples path and metadata.
\end{ldescription}
\end{Arguments}
\inputencoding{utf8}
\HeaderA{peakVisUI}{@title peakVisUI}{peakVisUI}
%
\begin{Description}\relax
@description This file contains all the code needed to run the peak
visualization from the metabolomics data analysis shiny page. The main
functions for constructing UI and server behavior are teh "peakVisUI" and
"peakVis" respectively. All of the other functions depend on
\end{Description}
%
\begin{Usage}
\begin{verbatim}
peakVisUI(id)
\end{verbatim}
\end{Usage}
%
\begin{Arguments}
\begin{ldescription}
\item[\code{id}] - character string used to map UI and server within App
\end{ldescription}
\end{Arguments}
\inputencoding{utf8}
\HeaderA{peakwidth\_est}{peakwidth\_est}{peakwidth.Rul.est}
%
\begin{Description}\relax
This function is designed to generate peak width estimates for
each TIC peak detected by sliding window analysis.
\end{Description}
%
\begin{Usage}
\begin{verbatim}
peakwidth_est(peak_vector, time, intensity, start = NULL, end = NULL,
  old_r2 = NULL)
\end{verbatim}
\end{Usage}
%
\begin{Arguments}
\begin{ldescription}
\item[\code{peak\_vector}] - numeric vector with names of specific time points of the
chromatography data measured. The numeric values correspond to indiciess
within the total chromatographic data that span the peak width.

\item[\code{time}] - This vector contains the time measurements during the
chromatography. This vector is used to match the values in peak\_vector
to the names in the intensity vector.

\item[\code{intensity}] - measured intensity values for chromatorgraphy

\item[\code{start}] - numeric index indicating where peak starts. Leave null.

\item[\code{end}] - same as above, leave null.

\item[\code{old\_r2}] - previous fit of model used to judge recursion of fit.
\end{ldescription}
\end{Arguments}
%
\begin{Details}\relax
This function takes in one peak vector at a time and runs a linear
model on the selected start and end points of a peak. By measuring the
change of the fit of the model, the function returns an index of values
corresponding to a peak. This function works recursively to estimate
the width of the peak. Ultimately, it returns the names of the final
points in the peak.
\end{Details}
%
\begin{Value}
This function returns a scalar value representing the estimated peak
width for a given peak.
\end{Value}
\inputencoding{utf8}
\HeaderA{peakwidth\_table}{peakwidth\_table}{peakwidth.Rul.table}
%
\begin{Description}\relax
This function is designed to generate estimates of peakwidth
for each peak within the peakList and some properties of each peak. After
this is done, the table of estimates is exported.
\end{Description}
%
\begin{Usage}
\begin{verbatim}
peakwidth_table(Autotuner, peakList, returned_peaks = 10)
\end{verbatim}
\end{Usage}
%
\begin{Arguments}
\begin{ldescription}
\item[\code{Autotuner}] - An Autotuner objected containing sample specific raw
data.

\item[\code{peakList}] - The list of data.frames containing information on each
peak observed across samples. This is the output generated by the
extract\_peaks functions that is required for this function to proceed.

\item[\code{returned\_peaks}] - A scalar number of peaks to return for visual
inspection. Five is the minimum possible value.
\end{ldescription}
\end{Arguments}
%
\begin{Details}\relax
The actual calculations used to estimate peakwidth are done within
the function "peakwidth\_est".
\end{Details}
%
\begin{Value}
This function will return a peak table with information on
the peak width for each detected peak across samples, the name
attribute for when the peak starts and ends, and the time points
associated with each of those parameters and for the midpoint.
\end{Value}
\inputencoding{utf8}
\HeaderA{peak\_time\_difference}{peak\_time\_difference}{peak.Rul.time.Rul.difference}
%
\begin{Description}\relax
This function is designed to return a data.frame containing info
on how
\end{Description}
%
\begin{Usage}
\begin{verbatim}
peak_time_difference(peak_table)
\end{verbatim}
\end{Usage}
%
\begin{Arguments}
\begin{ldescription}
\item[\code{peak\_table}] - table of peak width values extracted with the function
peak\_width\_table.
\end{ldescription}
\end{Arguments}
%
\begin{Details}\relax
This function is designed to determine what are the retention time
differences between peaks that are effectively the same between samples.
The similarity in peaks is determined by a threshold in retention time
similarity between peaks. This function returns the max peak width between
samples, and the time difference between peaks across samples in a data frame
object. The current and next row indexes are given to go back to the
peaktable object to plot peaks.
\end{Details}
%
\begin{Value}
This function returns a data.frame of peaks matched over time.
\end{Value}
\inputencoding{utf8}
\HeaderA{plot\_peaks}{@title plot\_peaks - this funciton plots the peak identified within chromatography.}{plot.Rul.peaks}
%
\begin{Description}\relax
This function plots individual peaks selected by signal
processing and expanded with a regression to allow the user to validate the
selected signal processing parameters.
\end{Description}
%
\begin{Usage}
\begin{verbatim}
plot_peaks(Autotuner, boundary = 10, peak, peak_difference, peak_table)
\end{verbatim}
\end{Usage}
%
\begin{Arguments}
\begin{ldescription}
\item[\code{Autotuner}] - An Autotuner objected containing sample specific raw
data.

\item[\code{boundary}] - UI input value that defines the boundary around the peak to
visualize it.

\item[\code{peak}] - Numeric index obtained from UI that indicates which peak
should be visualized.

\item[\code{peak\_difference}] - A data.frame containing peak overlap information.

\item[\code{peak\_table}] - A data.frame containing the sample specific peak
information.
\end{ldescription}
\end{Arguments}
%
\begin{Value}
This function outputs plots that are meant to go into the peakVis
UI.
\end{Value}
\inputencoding{utf8}
\HeaderA{plot\_signals}{plot\_signals - this funciton plots the peak identified within chromatography.}{plot.Rul.signals}
%
\begin{Description}\relax
This function plots the chromatography and the matching sliding
window signal processing results for each sample. Signal processing
functions will be the same color as the chromatography spectra, just a
lighter shade and a different type of line. The chromatography will be a
solid line, the signal will be a dashed line (lty = 2) with a .5 alpha,
and the thersholds will be dotted lines (lty = 3) with alpha values of 3.
\end{Description}
%
\begin{Usage}
\begin{verbatim}
plot_signals(Autotuner, threshold, sample_index, signals)
\end{verbatim}
\end{Usage}
%
\begin{Arguments}
\begin{ldescription}
\item[\code{Autotuner}] - MSest object created following Create\_MSest()
initialization function.

\item[\code{threshold}] - User input scalar value for the number of standard
deviations required to consider a peak to be significant.

\item[\code{sample\_index}] - which of all of the samples should the user plot.
Entered in as a numerical index value with length 1.

\item[\code{signals}] - A vector containing information on where signals are
located between samples.
\end{ldescription}
\end{Arguments}
%
\begin{Value}
Plots signal
\end{Value}
\inputencoding{utf8}
\HeaderA{signalProcessing}{signalProcessing}{signalProcessing}
%
\begin{Description}\relax
This function contains the server component of the signal
processing portion of the shiny app used within Autotuner.
\end{Description}
%
\begin{Usage}
\begin{verbatim}
signalProcessing(input, output, session, Autotuner)
\end{verbatim}
\end{Usage}
%
\begin{Arguments}
\begin{ldescription}
\item[\code{input}] - List of UI entered values.

\item[\code{output}] - Values exported from server to the UI after computation.

\item[\code{session}] - String used to match the namespace of the UI and server

\item[\code{Autotuner}] - An Autotuner objected containing sample specific raw
data.
\end{ldescription}
\end{Arguments}
\inputencoding{utf8}
\HeaderA{signalProcessingUI}{signalProcessingUI}{signalProcessingUI}
%
\begin{Description}\relax
This function contains the UI component of the signal processing
portion of the shiny app used within Autotuner.
\end{Description}
%
\begin{Usage}
\begin{verbatim}
signalProcessingUI(id)
\end{verbatim}
\end{Usage}
%
\begin{Arguments}
\begin{ldescription}
\item[\code{id}] - character string used to map UI and server within App
\end{ldescription}
\end{Arguments}
\inputencoding{utf8}
\HeaderA{ThresholdingAlgo}{ThresholdingAlgo}{ThresholdingAlgo}
%
\begin{Description}\relax
This function performs a sliding window analysis on the chromatograms
in order to identify peaks within the data. I would recommend to keep influence
low in order to use adjacent peak lengths as a measure of peak width.
\end{Description}
%
\begin{Usage}
\begin{verbatim}
ThresholdingAlgo(y, lag, threshold, influence)
\end{verbatim}
\end{Usage}
%
\begin{Arguments}
\begin{ldescription}
\item[\code{y}] - numerical vector of measured chromatographic intensity values

\item[\code{lag}] - scalar value of number of observations to calculate intensity
prior to peak selection.

\item[\code{threshold}] - number of standard deviations above chromatogram. Used
to detect significantly observed peaks.

\item[\code{influence}] - scalar values between 0-1 that describes how much the value
of a peak (measured index value above threshold) should contribute to the
sliding window analysis of downstream peaks.
\end{ldescription}
\end{Arguments}
\inputencoding{utf8}
\HeaderA{TIC\_params}{TIC\_params}{TIC.Rul.params}
%
\begin{Description}\relax
This function is designed to return the parameters related to
chromatography gathered during the TIC peak selection steps. An estimate is
given for maximum and minimum peak width as well as the bandwidth parameter
used in grouping.
\end{Description}
%
\begin{Usage}
\begin{verbatim}
TIC_params(peak_table, peak_difference)
\end{verbatim}
\end{Usage}
%
\begin{Arguments}
\begin{ldescription}
\item[\code{peak\_table}] - A data.frame containing information on peak width values
extracted with the function peak\_width\_table.

\item[\code{peak\_difference}] - A data.frame containing information on retention
time differences between peaks.
\end{ldescription}
\end{Arguments}
%
\begin{Value}
Returns a set of parameters to run xcms.
\end{Value}
\printindex{}
\end{document}
